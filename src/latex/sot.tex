\thispagestyle{empty}
\subsection{Single Object Tracking}

Single Object Tracking (SOT) развивался на протяжении многих лет, опираясь на достижения в области технологий и появления методов машинного обучения. В этом разделе приведён исторический обзор с классификацией подходов на этапы до использования машинного обучения, методы мелкого обучения и методы глубокого обучения.

\subsection{Ранние методы (1990-е): Геометрические подходы}
В 1990-х годах основными подходами для SOT являлись \textbf{генеративные методы}, использующие сильные геометрические допущения.

\begin{itemize}
    \item \textbf{Оптический поток (TLD)}: Основной метод, основанный на оценке движения пикселей на изображении.
    \begin{itemize}
        \item \href{http://vision.deis.unibo.it/ssalti/papers/Salti-TIP12.pdf}{TLD}
    \end{itemize}
    \item \textbf{Фильтр Калмана}: Прогнозирует следующее положение объекта, учитывая шум наблюдения и движения.
    \begin{itemize}
        \item \href{https://www.sciencedirect.com/science/article/pii/S1534580701000363}{Фильтр Калмана}
    \end{itemize}
    \item \textbf{Частичный фильтр}: Применяется для обработки неопределённости при отслеживании.
    \item \textbf{Ручные признаки (SIFT, SURF)}: Извлечение геометрических признаков вручную.
    \begin{itemize}
        \item \href{https://www.cs.ubc.ca/~lowe/papers/ijcv04.pdf}{SIFT}, \href{https://www.vision.ee.ethz.ch/en/publications/papers/articles/eth_biwi_00378.pdf}{SURF}
    \end{itemize}
\end{itemize}

\subsection*{Переход к методам обучения (2000-е)}
С 2000-х годов \textbf{методы на основе обучения} стали основными, разделившись на мелкое и глубокое обучение.

\subsection*{Методы неглубокого обучения}
Подходы мелкого обучения комбинировали методы машинного обучения с ручным извлечением признаков для отслеживания объектов.

\begin{itemize}
    \item \textbf{Метод опорных векторов (SVM)}: Использовался для бинарной классификации, где цель считалась положительным классом, а окружающая среда — отрицательным.
    \begin{itemize}
        \item \href{https://arxiv.org/pdf/1305.7352.pdf}{Метод SVM для отслеживания}
    \end{itemize}
    \item \textbf{Ансамблевое обучение}: Применение нескольких моделей для улучшения точности отслеживания.
    \begin{itemize}
        \item \href{https://arxiv.org/pdf/1706.07809.pdf}{Ансамблевое обучение в отслеживании}
    \end{itemize}
    \item \textbf{Разрежённое кодирование и корреляционные фильтры}: Применение разрежённых представлений для эффективного отслеживания.
    \begin{itemize}
        \item \href{https://www.sciencedirect.com/science/article/pii/S0031320317301141}{Разрежённое кодирование}, \href{https://arxiv.org/pdf/1707.00469.pdf}{Корреляционные фильтры}
    \end{itemize}
\end{itemize}

\subsection{Эра глубокого обучения}
Технологии глубокого обучения радикально изменили SOT, предложив более устойчивые и точные решения.

\begin{itemize}
    \item \textbf{Полносверточные сети (FCN) и корреляционные фильтры}: Ранние модели комбинировали FCN и корреляционные фильтры для предсказания объектов.
    \begin{itemize}
        \item \href{https://arxiv.org/pdf/1506.01497.pdf}{FCN для отслеживания}, \href{https://arxiv.org/pdf/1405.6569.pdf}{Корреляционные фильтры}
    \end{itemize}
    \item \textbf{Сиамские сети}: Использование парных сетей для сравнения объекта в текущем кадре с будущими кадрами.
    \begin{itemize}
        \item \href{https://arxiv.org/pdf/1708.07669.pdf}{Сиамские сети}, \href{https://arxiv.org/pdf/1710.08833.pdf}{Сиамские сети для видео}
    \end{itemize}
    \item \textbf{Рекуррентные нейронные сети (RNN)}: Моделирование временных зависимостей для отслеживания в видеопоследовательностях.
    \begin{itemize}
        \item \href{https://arxiv.org/pdf/1907.12838.pdf}{RNN для отслеживания}
    \end{itemize}
\end{itemize}

\subsection{Современные достижения в SOT}
Современные достижения включают интеграцию графовых нейронных сетей (GNN) и трансформеров для улучшения точности отслеживания.

\begin{itemize}
    \item \textbf{STARK}: Применяет модели на основе трансформеров для устойчивого отслеживания.
    \begin{itemize}
        \item \href{https://arxiv.org/pdf/2103.17154.pdf}{STARK}
    \end{itemize}
    \item \textbf{Keeptrack}: Метод, поддерживающий высокую точность отслеживания в сложных условиях.
    \begin{itemize}
        \item \href{https://arxiv.org/pdf/2010.05781.pdf}{Keeptrack}
    \end{itemize}
    \item \textbf{CSWinTT}: Современная модель, использующая механизмы внимания трансформеров для отслеживания объектов.
    \begin{itemize}
        \item \href{https://arxiv.org/pdf/2112.02759.pdf}{CSWinTT}
    \end{itemize}
\end{itemize}

\subsection{Эксперименты}
\begin{itemize}
    \item \textbf{Фильтр Калмана с TLD}: Начните с воспроизведения классических методов на основе геометрии, используя оптический поток для отслеживания движения объектов. Интегрируйте фильтр Калмана для сглаживания траектории.
    \item \textbf{SVM для мелкого обучения}: Реализуйте SVM-классификаторы на наборах данных с богатыми признаками для понимания подходов мелкого обучения.
    \item \textbf{Полносверточные сети + корреляционные фильтры}: Экспериментируйте с FCN в паре с корреляционными фильтрами для предсказания объектов.
    \item \textbf{Сиамские сети для реального времени}: Постройте и обучите сиамскую сеть для изучения встраиваний объектов и их отслеживания в последовательных кадрах.
    \item \textbf{Трансформеры для отслеживания}: Исследуйте модели на основе трансформеров, такие как STARK, для экспериментов с последними достижениями в SOT.
\end{itemize}